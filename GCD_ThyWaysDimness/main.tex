% ©2023 Alonso Madrigal Hernandez. Todos los derechos reservados. Este código es una creación original de Alonso Madrigal Hernandez y está protegido por las leyes de derechos de autor. %

\documentclass[12pt]{article}
\usepackage{graphicx} % Para imágenes
\usepackage[spanish]{babel} % Idioma del documento
\usepackage[colorlinks]{hyperref} % Para los enlaces
\usepackage{fancyhdr} % Para encabezados y pies de página
\usepackage{lastpage}
\usepackage{titlesec} % Control sobre los títulos
\usepackage{color} % Colores
\usepackage{geometry} % Margenes
\usepackage{afterpage}
\usepackage{setspace} % Espaciado entre lineas
\usepackage[dvipsnames]{xcolor}
\usepackage[nopatch=footnote]{microtype}
\usepackage{comment}
\usepackage{xspace}
\usepackage{fontspec}
\usepackage{amsmath,amsfonts,amssymb}
\usepackage{tikz}
\usetikzlibrary{shapes.geometric,fit} % Add this line to include geometric shapes

\usepackage{multicol}
\usepackage[toc,lof,lot]{multitoc}

\usepackage{enumitem}
\usepackage{booktabs}
\usepackage{float}
\usepackage{tablefootnote}
\usepackage{xhfill}
\usepackage{todonotes}
\usepackage{csquotes}

% \usepackage{plex-otf}
\usepackage[sfdefault]{plex-sans}
\usepackage{plex-serif}
\usepackage{plex-mono}
% \renewcommand*\familydefault{\sfdefault}
\usepackage[backend=biber,sorting=none]{biblatex} % Para referencias
\addbibresource{references.bib} % Archivo de referencias
% Configuración de márgenes
\geometry{
 a4paper,
 left=25mm,
 right=25mm,
 top=20mm,
 bottom=20mm
}

% Configuración del enlace de colores
\hypersetup{
    linkcolor=tocColor
    ,citecolor=Green
    ,filecolor=Mulberry
    ,urlcolor=NavyBlue
    ,menucolor=BrickRed
    ,runcolor=Mulberry
    ,linkbordercolor=BrickRed
    ,citebordercolor=Green
    ,filebordercolor=Mulberry
    ,urlbordercolor=NavyBlue
    ,menubordercolor=BrickRed
    ,runbordercolor=Mulberry
}


\newcommand\ruleafter[1]{#1\ \hrulefill}
\titleformat{\section}[hang]{\ttfamily\itshape\Large\bfseries\raggedright}{\thesection}{2pc}{\ruleafter}

% \newcommand\ruleafter[1]{\parbox{\linewidth}{\centering #1\ \rule[0.5ex]{5em}{0.4pt}}}
% \titleformat{\section}[hang]{\ttfamily\itshape\Large\bfseries\raggedright}{\thesection}{2pc}{\ruleafter}

\titlespacing*{\section}{0pt}{0pt}{8pt}

% Configuración de párrafos
% \setlength{\parindent}{15pt} % Sangría
\setlength{\parindent}{2.5em}
\setlength{\parskip}{1em} % Espaciado entre párrafos

% Encabezados y pies de página
\pagestyle{fancy}
\fancyhf{}
\fancyhead[L]{\color{textColor}\ttfamily\itshape GDD - \gameTitle}
\fancyhead[R]{\color{textColor}\ttfamily\itshape Jesús Jiménez Montero}

\fancyfoot[L]{\color{textColor}\ttfamily\itshape\raggedright\parbox[t]{0.4\textwidth}{\leftmark}}
\fancyfoot[C]{\color{textColor}\hfill\makebox[0pt][c]{\hyperlink{toc}{\itshape Volver al Índice}}\hfill}
\fancyfoot[R]{\color{textColor}\ttfamily\itshape\hfill Página \thepage \xspace de \pageref*{LastPage}}


\setlength{\headheight}{14.5pt}

% \renewcommand{\headrulewidth}{0.4pt}
\renewcommand{\footrulewidth}{0.4pt}

\newcommand{\gameTitle}{\texttt{\textbf{Thy Ways Dimness}}\xspace}
\newcommand{\hole}{\texttt{\textit{Hole}}\xspace}
\newcommand{\gunpowder}{\texttt{\textit{Gunpowder}}\xspace}
\newcommand{\humanity}{\texttt{\textit{Humanity}}\xspace}
\newcommand{\hunters}{\texttt{\textit{Hunters}}\xspace}
\newcommand{\hunter}{\texttt{\textit{Hunter}}\xspace}
\newcommand{\TWD}{\texttt{\textbf{TWD}}\xspace}

\newcommand{\fodder}{\textit{Fodder}\xspace}
\newcommand{\ranger}{\textit{Rangers}\xspace}
\newcommand{\specials}{\textit{Specials}\xspace}
\newcommand{\shield}{\textit{Shielders}\xspace}
\newcommand{\darkneer}{\textit{Darkneers}\xspace}

% For subsections
\titleformat{\subsection}[hang]{\large\bfseries}{\thesubsection}{1em}{}\titlespacing*{\subsection}{1em}{*1}{*1}

% For subsubsections
\titleformat{\subsubsection}[hang]{\normalsize\bfseries}{\thesubsubsection}{1em}{}\titlespacing*{\subsubsection}{2.5em}{*1}{*1}

\definecolor{pageColor}{RGB}{250,248,245}
\definecolor{textColor}{RGB}{61, 61, 61}
\definecolor{tocColor}{RGB}{9, 138, 177}
% Configuración del documento

\setlength{\marginparwidth}{2cm}

\renewcommand*{\multicolumntoc}{2}
\renewcommand*{\multicolumnlof}{2}
\renewcommand*{\multicolumnlot}{2}

% \setlength{\columnseprule}{1pt}

%%%%%%%%%%%%%%%%%%%%%%%%%%%%%%%%%%%%%%%%%%%%%%%%%%%%%%%%%%%%%%%%%%%%%%%%%%%%%%%%%%%%%%%%%%%%%%%%%
\begin{document}
\pagecolor{pageColor}
\color{textColor}
% Portada
\begin{titlepage}
    \vspace*{\fill}

    \centering
    \parbox{0.8\textwidth}{    % Logo de la universidad
        \includegraphics[width=0.7\textwidth]{Imagenes/marca-uji-color-fons-transparent.png}\par\vspace{1cm}

        {\Huge \bfseries GDD \\ \textit{Game Design Document} \par}
        {\Large \bfseries \gameTitle \par}

        % Información adicional
        \textsc{\large }
        \vspace{0.5cm} \\
        \textsc{\Large VJ1222 - \textit{Diseño Conceptual de Videojuegos}}
        \vspace{0.5cm} \\
        \textsc{\large Año: 2024 - 2025}
        \vfill

        % Nombres de los autores
        \textbf{Realizado por:}         \\
        \href{https://www.richardotomislav.com/}{Jesús Jiménez Montero}      \\
    }
    \vspace*{\fill}
\end{titlepage}

% Abstract
\begin{abstract}
    % TODO: Alargar el abstract un poco más
    This document contains both the One Sheet and the GDD for the conceptualizaction of \textit{Thy Ways Dimness}, an FPS, pseudo-horror roguelike.
\end{abstract}
{\small \textbf{Keywords} \textit{Thy Ways Dimness,} FPS, Roguelike, Horror, Pseudo-Horror, Game Design, GDD, One Sheet}
\newpage

% Índice de contenidos
%change the table of contents title
% \renewcommand{\contentsname}{Indice del documento}
\hypertarget{toc}{}

\tableofcontents

\newpage

% \renewcommand{\listtablename}{Lista de }
\listoftables
\newpage

% Índice de figuras
\listoffigures
\newpage

%%%%%%%%%%%%%%%%%%%%%%%%%%%%%%%%%%%%%%%%
% BEGINNING
%%%%%%%%%%%%%%%%%%%%%%%%%%%%%%%%%%%%%%%%

% \section*{\huge{One Sheet}}
\part{One Sheet}
% \section*{\huge{One Sheet}}
% \addcontentsline{toc}{section}{One Sheet}
\begin{multicols}{2}


\section{Título del Juego}
Thy Ways Dimness \footnote{\href{https://dictionary.cambridge.org/es/diccionario/ingles/dimness}{Meaning of dimness}}

%%%%%%%%%%%%%%%%%%%%%%%%%%%%%%%%%%%%%%%%
%%%%%%%%%%%%%%%%%%%%%%%%%%%%%%%%%%%%%%%%

\section{Público objetivo}
The game is intentded for players that like wave-based games, with a focus on room clearing and fast-paced combat.
d
Using \textbf{Bartle's Taxonomy}, the game focuses primarly on two types of players. \textbf{Killers}, who will enjoy the killing aspect of clearing full of enemies and \textbf{Achievers} because the game will include achivements and (to be considered) collectibles, which could include the \textbf{Explorers} type of player.

Also, the \textbf{Socializers} would be left of the equation, however, the game could support some kind of coop multiplayer, or even \textit{versus} mode.

%%%%%%%%%%%%%%%%%%%%%%%%%%%%%%%%%%%%%%%%
%%%%%%%%%%%%%%%%%%%%%%%%%%%%%%%%%%%%%%%%


\section{Resumen de la historia}
The game is set on a dark-medieval like world, where all human life has become extinct due to a event. This event was that the Earth crust collapsed, opening a big \hole that opened the gates to dark creatures.

The creatures made of pure darkenss were mostly indestructible, except for a big weakness: \textbf{Light} and \textbf{Fire}; Light would expose them, and revealing their \textit{“flesh”} and be able to be damaged, then by using weapons covered in fire, these would be able to die. \humanity's will to live made them forcibly discover \gunpowder, which combined both elements into \humanity's last reasource.

\humanity was able to persevere by moving their cities near the \hole; as for some reason, as closer \humanity was to the \hole, the less the creatures would attack them. Thus, the villages near the \hole established and the legion of \hunters was created.

As \gunpowder was limited, only the best warriors were assigned to the \gunpowder weapons; and to save \humanity only one thing can be done; go down to the \hole.

The player is one of those \hunters, and you need to kill everything down there and save them all.


%%%%%%%%%%%%%%%%%%%%%%%%%%%%%%%%%%%%%%%%
%%%%%%%%%%%%%%%%%%%%%%%%%%%%%%%%%%%%%%%%

\section{Modos de juego}
The game would have two modes of gameplay; \textbf{Story Mode} and \textbf{Endless Mode}.

The first being a chapter-based structure where the player would descend into the \hole further and further until defeating a boss or solving the \hole \textit{problem}.

The second is a wave-based mode where the player would survive as long as possible; appart from being used as a leaderboard and achievment hunter gamemode, it would be used to practice \textit{(learn new weapons, enemies weakness, etc.)}.

%%%%%%%%%%%%%%%%%%%%%%%%%%%%%%%%%%%%%%%%
%%%%%%%%%%%%%%%%%%%%%%%%%%%%%%%%%%%%%%%%

\section{¿Porque es este juego original?}
The main selling point of the game is the extreme use of darkness and light. First, the game is \textit{VERY} dark, and enemies cannot be located by eyesight until they are close, following their faintly glowing eyes, or locating them by sound. One simple way to see them is by using \textbf{flares} which are auto-reloaded but only have a few ready to use and fade out with time.

However, enemies can also be revealed by shooting them. As \\gunpowder is special in this world, the bullets will permanently reveal the enemy; also, if the player missed and bullets fly by the enemy, this will reveal them but only for a short time \textit{(similar fade out to flares)}.

This is the main mechanic that separates this game from others.

\section{¿Y qué es lo que lo hace interesante?}
How the game uses darkness to defeat enemies is something that hasn't been explored much in other games.

%%%%%%%%%%%%%%%%%%%%%%%%%%%%%%%%%%%%%%%%
%%%%%%%%%%%%%%%%%%%%%%%%%%%%%%%%%%%%%%%%
\section{¿Cuales son las principales diferencias con juegos del mismo género y cuáles son?}
The darkness is used in many games, but not as a main mechanic. Lightning is commonly used to obscure the game and not being able to see enemies in other games.

\end{multicols}
\newpage

\part{Game Design Document}
% \section*{\huge{Game Design Document}}
% \addcontentsline{toc}{section}{Game Design Document}
\section{One Sheet}
Contains the One Sheet with basic information about the video game, especially the Special Selling Points.

\section{Player Design}
Develop the following points:
\subsection{Sociocultural Characteristics}
Sociocultural characteristics of the video game.
\subsection{Gender Issues}
Possible gender issues.
\subsection{Player Types}
Design for Killers, Achievers, Socializers, and Explorers.

\section{Game Rules}
Develop the characteristics of the rules in relation to:
\subsection{Objectives}
\subsection{Limits}
\subsection{Players}
\subsection{Obstacles and Conflicts}
\subsection{Foundational Rules}
\subsection{Operational Rules}
\subsection{Written Rules}
\subsection{Behavioral Rules}
\subsection{Competition Rules}
\subsection{Gameplay Tips}

\section{Mechanics}
Correct and detailed description of mechanics:
\subsection{Character Mechanics}
\subsection{NPC Mechanics}
\subsection{Core Mechanics}
\subsection{Mechanics Table}
Mechanics Table includes:
\subsubsection{Description}
\subsubsection{Attributes}
\subsubsection{Dynamics and Actions}
\subsubsection{Triggers}
\subsubsection{Resources}

\section{Game Balance}
\subsection{Transitive Relationships}
\subsection{Intransitive Relationships}
\subsection{Unfair Moments}
Evaluation of whether there may be unfair moments.
\subsection{Difficulty Evolution}
Description of the evolution of difficulty.

\section{Gameplay}
Development of the challenge structure:
\subsection{Atomic Challenges}
\subsection{Intermediate Challenges}
\subsection{Final Challenge}

\section{Level Design}
\subsection{Beat Chart}
\subsection{Characteristics}
Characteristics include space, resources, shapes, information, challenges, rhythm, integrated narrative, aesthetics.
\subsection{Sketches}

\section{Sound Design}
\subsection{Soundtrack Design}
\subsection{Ambient Sound Design}
\subsection{Mechanics Sound Design}

\section{Gamification Design}
\subsection{Objectives}
\subsection{Conflicts}
\subsection{Mechanics}

\section{Monetization Design}

\section{Multiplayer Design}

\end{document}


\newpage
% \nocite{*}
\todo{QUITAR NOCITE PARA LA BIBLIOGRAFÍA}
\printbibliography % Imprimir la bibliografía
\end{document}