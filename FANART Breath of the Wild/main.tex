\documentclass{cup-pan}
\usepackage{blindtext}
\usepackage{comment}
\usepackage{setspace}
\spacing{1.15}
\renewcommand*\contentsname{Summary of the contents}

\title{Practical Coursework,\\ Video Game FanArt, \\ The Legend of Zelda: Breath of the Wild.}

\author{Jesus Jimenez Montero}

\affil[1] {Informática Básica, VJ1202}
\affil[2] {Expresión Artística, VJ1204}
\affil[3] {Inglés Moderno, VJ1205}

%% Corresponding author
\corrauthor{Jesús Jiménez Montero}

%% Abbreviated author list for the running footer
\runningauthor{Jesus JM}

\addbibresource{refs.bib}

\begin{document}

\maketitle
%\hrule
\textcolor{PANDarkBlue}{\hrule}

\tableofcontents
\newpage

\begin{abstract}
This document will be written in English due to the collaboration with the subject, "Inglés Moderno". 

\end{abstract}

\section{Introduction}

    \textcolor{PANDarkBlue}{\large Description of the game}\\
    
    The Legend of Zelda, Breath of the Wild; is a video game exclusive to the Nintendo Switch console, released in 3rd March 2017. 
    Breath of the Wild is an open-world adventure game based on the world of Hyrule, 100 years after the battle that took place where Ganon was sealed in the castle of Hyrule with Zelda protecting the seal. Link is almost defeated on this battle and goes to a deep sleep.
    
    After 100 years as previously mentioned, he's woken up by a distant Zelda from it's dormancy and goes out of the cave he was staying at, greeted by a distant and sisnister Hyrule Castle. And the adventure can began!\\
    
    \textcolor{PANDarkBlue}{\large Adaptation of the hero's quest}\\
    One of the requirements for this coursework was to adapt the Hero's Quest to the drawings.
    To make my drawings accomplish this task, I used the key points of Link's Adventure, being:\\
        \begin{enumerate}
            \item After the extended tutorial where Link's gets its paraglider to get down the plateau where the game begins and one of the first things the player does is look to the Dueling Peaks, one of the key points on the game, where Link discovers its true quest.\\
            \item Where the player gains its confidence to destroy a Guardian, whose quest is to keep out intruders from key areas. This also a key point of the game because destroying a guardian means you are one step closer to defeating Ganon. \\
            \item ESTE PUNTO HACE FALTA COMPLETARLO YA QUE NO TENGO NINGUN DIBUJO \\
        \end{enumerate}
    
\section{Reference images}

    
    
\section{Analysis of formal elements of a concept art}
\section{First Illustration}
\section{Second Illustration}
\section{Third Illustration}
\section{Conclusions}


\newpage
\section{Example of a first section}
\label{sec:overview}

Here's an example parenthetical citation \citep{lees2010theoretical} and a text citation: \citet{urmson2008autonomous}. There isn't a good, up-to-date BibTeX style for the Chicago style, so we're using \texttt{biblatex-chicago} instead. This means you'll need to run \texttt{biber} instead of \texttt{bibtex} if you're compiling this template on your local \LaTeX{} installation: on Overleaf, \texttt{biber} is run automatically. You can add pre-notes with citations \citep[see also][]{urmson2008autonomous} too, as well as multiple citations \citep{geiger2012we,leesother} in a single \verb|\citep{...}| or \verb|\citet{...}|. 

This is an equation, numbered
\begin{equation}
l(\Lambda)=\sum_{i=1}^{n} \sum_{w=1}^{q} (z_{i w} \ln (\lambda_{i w}) - \lambda_{i w} - \ln (z_{i w}!))
\label{eq:poisson}
\end{equation}
and one that is not numbered
\begin{equation*}
\int_0^{+\infty}e^{-x^2}dx=\frac{\sqrt{\pi}}{2}
\end{equation*}
and one inlined: $e^{i\pi}=-1$. As usual you can cross-reference equations with Equation \ref{eq:poisson} or \eqref{eq:poisson}.

Figure \ref{fig:example} shows a normal figure, while figure \ref{fig:twosubs} show one made up of two sub-figures. Figure \ref{fig:landscape} is an example of a landscaped figure. You can use the \verb|\subcaption{...}| command from the \texttt{subcaption} package to add captions for subfigures and subtables, but do not use the \texttt{subfigure} package: it is incompatible with this template.

\begin{figure}[bt]
\centering
\includegraphics[width=0.6\textwidth]{example-image}
\caption{This is a figure caption. Let's see what happens when it's long and contains citations \citep{geiger2012we} and cross-references: \ref{sec:overview}. Yep, works. Captions should not contain manual line breaks!}
\label{fig:example}
\end{figure}

\begin{table}[bt]
\caption{Automobile Land Speed Records (GR 5-10). Source: \url{https://www.sedl.org/afterschool/toolkits/science/pdf/ast_sci_data_tables_sample.pdf}}
\label{tab:example}
\centering
\begin{tabular}{l l l l r}
\headrow \thead{Speed (mph)} & \thead{Driver} & \thead{Car} & \thead{Engine} & \thead{Date} \\
407.447     & Craig Breedlove & Spirit of America          & GE J47    & 8/5/63   \\
413.199     & Tom Green       & Wingfoot Express           & WE J46    & 10/2/64  \\
434.22      & Art Arfons      & Green Monster              & GE J79    & 10/5/64  \\
468.719     & Craig Breedlove & Spirit of America          & GE J79    & 10/13/64 \\
526.277     & Craig Breedlove & Spirit of America          & GE J79    & 10/15/65 \\
536.712     & Art Arfons      & Green Monster              & GE J79    & 10/27/65 \\
555.127     & Craig Breedlove & Spirit of America, Sonic 1 & GE J79    & 11/2/65  \\
576.553     & Art Arfons      & Green Monster              & GE J79    & 11/7/65  \\
600.601     & Craig Breedlove & Spirit of America, Sonic 1 & GE J79    & 11/15/65 \\
622.407     & Gary Gabelich   & Blue Flame                 & Rocket    & 10/23/70 \\
633.468     & Richard Noble   & Thrust 2                   & RR RG 146 & 10/4/83  \\
763.035     & Andy Green      & Thrust SSC                 & RR Spey   & 10/15/97\\
\end{tabular}

\end{table}


Lorem ipsum dolor sit amet, consectetur adipiscing elit, sed do eiusmod tempor incididunt ut labore et dolore magna aliqua. Ut enim ad minim veniam, quis nostrud exercitation ullamco laboris nisi ut aliquip ex ea commodo consequat. Duis aute irure dolor in reprehenderit in voluptate velit esse cillum dolore eu fugiat nulla pariatur. Excepteur sint occaecat cupidatat non proident, sunt in culpa qui officia deserunt mollit anim id est laborum.

Ut enim ad minima veniam, quis nostrum exercitationem ullam corporis suscipit laboriosam, nisi ut aliquid ex ea commodi consequatur? Quis autem vel eum iure reprehenderit, qui in ea voluptate velit esse, quam nihil molestiae consequatur, vel illum, qui dolorem eum fugiat, quo voluptas nulla pariatur?


Ut enim ad minima veniam, quis nostrum exercitationem ullam corporis suscipit laboriosam, nisi ut aliquid ex ea commodi consequatur? Quis autem vel eum iure reprehenderit, qui in ea voluptate velit esse, quam nihil molestiae consequatur, vel illum, qui dolorem eum fugiat, quo voluptas nulla pariatur?


\begin{figure}
\begin{minipage}{0.47\textwidth}
\includegraphics[width=\linewidth]{example-image}
\subcaption{This is a subfigure}
\end{minipage}
\hfill
\begin{minipage}{0.47\textwidth}
\includegraphics[width=\linewidth]{example-image}
\subcaption{This is another subfigure}
\end{minipage}

\caption{This is a caption for the entire figure}
\label{fig:twosubs}
\end{figure}

\begin{sidewaysfigure}
\centering
\includegraphics[width=19cm]{example-image}
\caption{This is a figure caption}
\label{fig:landscape}
\end{sidewaysfigure}

At vero eos et accusamus et iusto odio dignissimos ducimus, qui blanditiis praesentium voluptatum deleniti atque corrupti, quos dolores et quas molestias excepturi sint, obcaecati cupiditate non-provident, similique sunt in culpa, qui officia deserunt mollitia animi, id est laborum et dolorum fuga. Et harum quidem rerum facilis est et expedita distinctio. Nam libero tempore, cum soluta nobis est eligendi optio, cumque nihil impedit, quo minus id, quod maxime placeat, facere possimus, omnis voluptas assumenda est, omnis dolor repellendus. Temporibus autem quibusdam et aut officiis debitis aut rerum necessitatibus saepe eveniet, ut et voluptates repudiandae sint et molestiae non-recusandae. Itaque earum rerum hic tenetur a sapiente delectus, ut aut reiciendis voluptatibus maiores alias consequatur aut perferendis doloribus asperiores repellat

At vero eos et accusamus et iusto odio dignissimos ducimus, qui blanditiis praesentium voluptatum deleniti atque corrupti, quos dolores et quas molestias excepturi sint, obcaecati cupiditate non-provident, similique sunt in culpa, qui officia deserunt mollitia animi, id est laborum et dolorum fuga. Et harum quidem rerum facilis est et expedita distinctio. Nam libero tempore, cum soluta nobis est eligendi optio, cumque nihil impedit, quo minus id, quod maxime placeat, facere possimus, omnis voluptas assumenda est, omnis dolor repellendus. Temporibus autem quibusdam et aut officiis debitis aut rerum necessitatibus saepe eveniet, ut et voluptates\footnote{Sed ut perspiciatis, unde omnis iste natus error sit voluptatem accusantium doloremque laudantium, totam rem aperiam eaque ipsa, quae ab illo inventore veritatis et quasi architecto beatae vitae dicta sunt, explicabo. Nemo enim ipsam voluptatem, quia voluptas sit, aspernatur aut odit aut fugit, sed quia consequuntur magni dolores eos, qui ratione voluptatem sequi nesciunt, neque porro quisquam est, qui dolorem ipsum, quia dolor sit amet consectetur adipisci[ng] velit, sed quia non-numquam [do] eius modi tempora inci[di]dunt, ut labore et dolore magnam aliquam quaerat voluptatem.} repudiandae sint et molestiae non-recusandae. Itaque earum rerum hic tenetur a sapiente delectus, ut aut reiciendis voluptatibus maiores alias consequatur aut perferendis doloribus asperiores repellat.

\blinddocument

\bigskip
If you have supplementary material (such as an appendix), it should not be included in your manuscript but uploaded as a separate PDF.

\paragraph*{Acknowledgments.} We are grateful for the technical assistance of A.~Author.

\paragraph*{Data Availability Statement.} A statement about how to access data, code and other materials allowing users to understand, verify and replicate findings — e.g. Replication data and code can be found in Harvard Dataverse: \url{https://doi.org/link}.



\printbibliography

\end{document}