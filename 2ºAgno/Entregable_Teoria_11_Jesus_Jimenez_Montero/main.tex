% !TeX TS-program = xelatex
\documentclass[12pt]{article}
\usepackage{listings}
\usepackage{float}
\usepackage{xcolor}
\usepackage{graphicx}
\usepackage{booktabs}
\usepackage{marginnote}
\usepackage{caption}
\usepackage{multicol}
\definecolor{codegreen}{rgb}{0,0.6,0}
\definecolor{codegray}{rgb}{0.5,0.5,0.5}
\definecolor{codepurple}{rgb}{0.58,0,0.82}
\definecolor{backcolour}{rgb}{0.95,0.95,0.92}
\lstdefinestyle{mystyle}{
	backgroundcolor=\color{backcolour},   commentstyle=\color{codegreen},
	keywordstyle=\color{magenta},
	numberstyle=\tiny\color{codegray},
	stringstyle=\color{codepurple},
	basicstyle=\ttfamily\footnotesize,
	breakatwhitespace=false,
	breaklines=true,
	captionpos=b,
	keepspaces=true,
	numbers=left,
	numbersep=5pt,
	showspaces=false,
	showstringspaces=false,
	showtabs=false,
	tabsize=2
}
%"mystyle" code listing set
\lstset{style=mystyle}
%\lstset{basicstyle=\ttfamily\footnotesize,breaklines=true}

\usepackage[breaklinks]{hyperref}
% Setup de hiperenlaces
\hypersetup{
	colorlinks=true,
	linkcolor=blue,
	filecolor=magenta,
	urlcolor=cyan,
	pdftitle={Apuntes de Consolas y Videojuegos},
	pdfpagemode=FullScreen,
	citecolor = green
}

\usepackage{url}

\usepackage{fontspec}
\setmainfont{AT-NameSansText}[
	Path=./NameSansStatic/,
	Extension = .otf,
	UprightFont=*-Regular,
	BoldFont=*-Bold,
	ItalicFont=*-RegularItalic,
	BoldItalicFont=*-BoldItalic,
	Numbers = OldStyle
]

\setmonofont{ATNameMono}[
	Path=./NameMonoStatic/,
	Extension = .otf,
	UprightFont=*-Regular,
	BoldFont=*-Bold,
	ItalicFont=*-RegularItalic,
	BoldItalicFont=*-BoldItalic
]

\usepackage[backend=biber]{biblatex}
\addbibresource{referencias.bib}
\usepackage{notoccite}
\renewcommand{\figurename}{\textbf{Figura.}}
\renewcommand{\tablename}{\textbf{Tabla.}}

\begin{document}
\nocite{namesans_about}
\nocite{namesans}
\nocite{namemono}


	\begin{titlepage}
		\begin{center}
			{\Huge \textbf{Entregable de Teoría 11}}

			\vspace{2cm}

			{\Large \textit{Jesús Jiménez Montero}}

			\vspace{2cm}
		\end{center}
	\end{titlepage}

	% ÍNDICE
	%\renewcommand{\tableofcontents}{Indice general}
	\newpage
	\renewcommand{\contentsname}{Tabla de contenidos}
	\setcounter{secnumdepth}{5}
	\tableofcontents
	\setcounter{tocdepth}{4}
	\newpage

%---------------------------------
%DOCUMENTO
%---------------------------------

\section{Caché total/completa-mente asociativa}

	\subsection{Resumen de correspodencia directa}

		La forma más sencilla de caché, se denomina correspondencia directa,
		la cual asigna una posición de la caché basándose en la dirección de
		memoria.
		Para poder alojar el contenido de varias posiciones diferentes de
		la memoria principal, se usan las etiquetas, las cuales llevan la información
		de la dirección necesaria para identificar si una palabra de la
		caché se corresponde con una palabra de la memoria principal.\footnote{\textit{Texto del entregable 10 de teoría}}

	\subsection{Caché total/completa-mente asociativa}

		Un bloque de memoria puede estar asociado a cualquiera de las entradas de la caché. Para encontrar un bloque determinado en una caché completamente asociativa, todas las entradas de la caché deben ser inspeccionadas, ya que un bloque puede encontrarse en cualquiera de ellas.

		Para que la búsqueda sea factible, se realiza en paralelo a través del comparador asociado a cada entrada de la caché. Estos comparadores incrementan significativamente el coste del hardware, haciendo que el emplazamiento completamente asociativo sea factible solo para cachés con un pequeño número de bloques. \cite{patterson-2011}\\

		Resumiendo, en este tipo de caché todos los bloques (o ranuras) de caché se tratan por igual y pueden ponerse en cualquier lugar de la caché. Esto también significa que no se interpreta ninguna de las direcciones como índices; dando lugar a todo sea una etiqueta. \cite{fully_asociative_cache}\\

		\newpage
		Esto significa que habría que buscar una dirección guardada en caché de forma \textbf{específica}.

		Funciona de la siguiente manera:
		\begin{enumerate}
			\item Se escribe la etiqueta y el dato en un bloque vacío. Si no hay ninguno...
			\item Se busca en toda la RAM si hay algún bloque de RAM con la misma TAG...
			\item Si no hay ningún bloque vacío o la etiqueta no se encuentra en RAM; se reemplaza el bloque \textbf{más antiguo} por el nuevo.
		\end{enumerate}

	\subsection{Escritura directa e retardada}

		Cuando se escriben datos en la RAM, la memoria guarda un valor distinto al de la caché, lo que da lugar a que la caché y memoria sean incoherentes. Para mantener esta coherencia, existen dos técnicas:

		\begin{itemize}
			\item \textbf{Escritura directa:} Para mantener esta coherencia entre memorias, simplemente se escriben los datos tanto en la memoria como en la caché. \cite{patterson-2011}
			\item \textbf{Escritura retardada:} Al contrario, este tipo de escritura actualiza los bloques de caché en primer lugar para después guardar el bloque en el nivel inferior de la jerarquía de memoria cuando se reemplace el bloque. \cite{patterson-2011}
		\end{itemize}


%---------------------------------
%BIBLIOGRAFÍA
%---------------------------------
\newpage

\printbibliography[title={Bibliografía y agradecimientos}]
\end{document}