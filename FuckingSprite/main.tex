\documentclass[12pt]{article}
\usepackage{graphicx} % Para imágenes
\usepackage[spanish]{babel} % Idioma del documento
\usepackage[colorlinks]{hyperref} % Para los enlaces
\usepackage{fancyhdr} % Para encabezados y pies de página
\usepackage{lastpage}
\usepackage{titlesec} % Control sobre los títulos
\usepackage{color} % Colores
\usepackage{geometry} % Margenes
\usepackage{afterpage}
\usepackage{setspace} % Espaciado entre lineas
\usepackage[dvipsnames]{xcolor}
\usepackage{xspace}
\hyphenpenalty=5000
\tolerance=1000
\usepackage{multicol}
% \usepackage[toc,lof,lot]{multitoc}
\usepackage{subfig}
\usepackage{float}



\usepackage{fontspec}
% \usepackage{plex-otf}
\usepackage[sfdefault]{plex-sans}
\usepackage{plex-serif}
\usepackage{plex-mono}
% \renewcommand*\familydefault{\sfdefault}
% \defaultfontfeatures{Renderer=Harfbuzz}
% \setmainfont[
%     ItalicFont=Labrada-Italic.ttf,
%     BoldFont=Labrada-Variable.ttf,
%     BoldItalicFont=Labrada-Italic.ttf,
%     RawFeature={+axis={wght=400}},
%     BoldFeatures={RawFeature={+axis={wght=700}}},
%     BoldItalicFeatures={RawFeature={+axis={wght=700}}}
% ]{Labrada-Variable.ttf}

% Configuración de márgenes
\geometry{
 a4paper,
 left=25mm,
 right=25mm,
 top=20mm,
 bottom=20mm
}

% Configuración del enlace de colores
\hypersetup{
    linkcolor=tocColor,
    citecolor=Green,
    filecolor=Mulberry,
    urlcolor=NavyBlue,
    menucolor=BrickRed,
    runcolor=Mulberry,
    linkbordercolor=BrickRed,
    citebordercolor=Green,
    filebordercolor=Mulberry,
    urlbordercolor=NavyBlue,
    menubordercolor=BrickRed,
    runbordercolor=Mulberry
}

\newcommand\ruleafter[1]{#1\ \hrulefill}
\titleformat{\section}[hang]{\itshape\Large\bfseries\raggedright}{\thesection}{2pc}{\ruleafter}
\titlespacing*{\section}{0pt}{0pt}{8pt}

% Configuración de párrafos
\setlength{\parindent}{2.5em}
\setlength{\parskip}{1em} % Espaciado entre párrafos

% Encabezados y pies de página
\pagestyle{fancy}
\fancyhf{}
\fancyhead[L]{\color{textColor}\itshape Practica de diseño de personajes animación SpriteSheet}
\fancyhead[R]{\color{textColor}\itshape Jesús Jiménez Montero}

\fancyfoot[L]{\color{textColor}\itshape\raggedright\parbox[t]{0.4\textwidth}{\leftmark}}
\fancyfoot[C]{\color{textColor}\hfill\makebox[0pt][c]{\hyperlink{toc}{\itshape Volver al Índice}}\hfill}
\fancyfoot[R]{\color{textColor}\itshape\hfill Página \thepage \xspace de \pageref*{LastPage}}

\setlength{\headheight}{14.5pt}
\renewcommand{\footrulewidth}{0.4pt}

% For subsections
\titleformat{\subsection}[hang]{\large\bfseries}{\thesubsection}{1em}{}\titlespacing*{\subsection}{1em}{*1}{*1}

% For subsubsections
\titleformat{\subsubsection}[hang]{\normalsize\bfseries}{\thesubsubsection}{1em}{}\titlespacing*{\subsubsection}{2.5em}{*1}{*1}

\definecolor{pageColor}{RGB}{250,248,245}
\definecolor{textColor}{RGB}{61, 61, 61}
\definecolor{tocColor}{RGB}{9, 138, 177}

\setlength{\marginparwidth}{2cm}

% \renewcommand*{\multicolumntoc}{2}
% \renewcommand*{\multicolumnlof}{2}
% \renewcommand*{\multicolumnlot}{2}

%%%%%%%%%%%%%%%%%%%%%%%%%%%%%%%%%%%%%%%%%%%%%%%%%%%%%%%%%%%%%%%%%%%%%%%%%%%%%%%%%%%%%%%%%%%%%%%%%
\begin{document}
\pagecolor{pageColor}
\color{textColor}
% Portada
\begin{titlepage}
    \vspace*{\fill}

    \centering
    \parbox{0.8\textwidth}{    % Logo de la universidad
        \includegraphics[width=0.7\textwidth]{imgs/marca-uji-color-fons-transparent.png}\par\vspace{1cm}

        {\Huge \bfseries \textit{Práctica 4 - LA2} \par}
        {\Large \bfseries Practica de diseño de personajes animación SpriteSheet \par}

        % Información adicional
        \textsc{\large }
        \vspace{0.5cm} \\
        \textsc{\Large VJ1223 - \textit{Arte del Videojuego}}
        \vspace{0.5cm} \\
        \textsc{\large Año: 2024 - 2025}
        \vfill

        % Nombres de los autores
        \textbf{Realizado por:}         \\
        \href{https://www.richardotomislav.com/}{Jesús Jiménez Montero}      \\
    }
    \vspace*{\fill}
\end{titlepage}

% Índice de contenidos
%change the table of contents title
% \renewcommand{\contentsname}{Indice del documento}
\hypertarget{toc}{}
\tableofcontents
\newpage

\listoffigures
\newpage


\section{Apartado 1 y Apartado 2}
    He usado el personaje que había en el Aula Virtual.
    \begin{figure}[H]
        \centering
        \includegraphics[height=0.8\textheight]{imgs/2D-Character_mio.png}
        \caption{Personaje estático sin modificar todavía}
        \label{fig:static}
    \end{figure}

\newpage
\section{Apartado 3}
    \begin{figure}[h]
        \centering
        \includegraphics[width=\textwidth]{imgs/2D-Character.png}
        \caption{Atlas del personaje}
        \label{fig:atlas}
    \end{figure}
\newpage
\section{Apartado 4}
    \begin{figure}[h]
        \centering
        \includegraphics[width=\textwidth]{imgs/spritesheet_og.png}
        \caption{Spritesheet de la animación}
        \label{fig:sprt_og}
    \end{figure}
    \begin{figure}[h]
        \centering
        \includegraphics[width=\textwidth]{imgs/spritesheet_res.png}
        \caption{SpriteSheet reescalada a 1000x1000 para Unity}
        \label{fig:sprt_res}
    \end{figure}
\newpage
\section{Apartado 5}
    La animación se puede ver en el \href{https://drive.google.com/open?id=10MmPTifcON0xG-mMmJiE0lPxJajcDE9M&usp=drive_fs}{enlace a mi carpeta de Drive}.
\section{Apartado 6}
    \begin{figure}[h]
        \centering
        \includegraphics[width=\textwidth]{spritesheet_unity.png}
        \caption{Captura de Unity del personaje animado}
        \label{fig:unity}
    \end{figure}

\end{document}