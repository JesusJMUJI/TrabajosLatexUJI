% !TEX program = xelatex
% ©2023 Alonso Madrigal Hernandez. Todos los derechos reservados. Este código es una creación original de Alonso Madrigal Hernandez y está protegido por las leyes de derechos de autor. %


\documentclass[12pt]{article}
\usepackage{graphicx} % Para imágenes
\usepackage[utf8]{inputenc} % Soporte UTF-8
\usepackage[spanish]{babel} % Idioma del documento
\usepackage[colorlinks]{hyperref} % Para los enlaces
\usepackage{fancyhdr} % Para encabezados y pies de página
\usepackage{titlesec} % Control sobre los títulos
\usepackage{color} % Colores
\usepackage{geometry} % Margenes
\usepackage{setspace} % Espaciado entre lineas
\usepackage[dvipsnames]{xcolor}
\usepackage{microtype}

% Configuración de márgenes
\geometry{
 a4paper,
 left=25mm,
 right=25mm,
 top=20mm,
 bottom=20mm
}

% Configuración del enlace de colores
\hypersetup{
    linkcolor=BrickRed
    ,citecolor=Green
    ,filecolor=Mulberry
    ,urlcolor=NavyBlue
    ,menucolor=BrickRed
    ,runcolor=Mulberry
    ,linkbordercolor=BrickRed
    ,citebordercolor=Green
    ,filebordercolor=Mulberry
    ,urlbordercolor=NavyBlue
    ,menubordercolor=BrickRed
    ,runbordercolor=Mulberry
}

\usepackage{fontspec}
\setmainfont{NeueHaasDisplay}[
    Path = ./fonts/,
    Extension = .ttf,
    UprightFont = {*Roman},
    ItalicFont = {*RomanItalic},
    BoldFont = {*Bold},
    BoldItalicFont = {*BoldItalic},
]

% Define a range of font weights manually without wildcard
\newfontface\Light{NeueHaasDisplayLight}[
    Path = ./fonts/,
    ItalicFont = NeueHaasDisplayLightItalic,  % Explicitly provide the full Italic font name
    Extension = .ttf
]

\newfontface\Medium{NeueHaasDisplayMedium}[
    Path = ./fonts/,
    ItalicFont = NeueHaasDisplayMediumItalic,
    Extension = .ttf
]

\newfontface\Thin{NeueHaasDisplayThin}[
    Path = ./fonts/,
    ItalicFont = NeueHaasDisplayThinItalic,
    Extension = .ttf
]

\newfontface\XThin{NeueHaasDisplayXThin}[
    Path = ./fonts/,
    ItalicFont = NeueHaasDisplayXThinItalic,
    Extension = .ttf
]

\newfontface\XXThin{NeueHaasDisplayXXThin}[
    Path = ./fonts/,
    ItalicFont = NeueHaasDisplayXXThinItalic,
    Extension = .ttf
]

\newfontface\Black{NeueHaasDisplayBlack}[
    Path = ./fonts/,
    ItalicFont = NeueHaasDisplayBlackItalic,
    Extension = .ttf
]

% Here are some other weights:

% \Light{This is light text.} \\
% \Medium{This is medium text.} \\
% \Thin{This is thin text.} \\
% \XThin{This is extra thin text.} \\
% \XXThin{This is double extra thin text.} \\
% \Black{This is black text.}

% Cambiar tipo de letra
% \renewcommand{\rmdefault}{ptm} % Cambia el tipo de letra a Times

% Configuración de títulos de secciones
\titleformat{\section}[hang]{\Large\bfseries}{\thesection}{2pc}{}
\titlespacing*{\section}{0pt}{0pt}{20pt}

% Configuración de párrafos
\setlength{\parindent}{15pt} % Sangría
\setlength{\parskip}{1em} % Espaciado entre párrafos

% Encabezados y pies de página
\fancyhf{}
\fancyhead[L]{\leftmark}
\fancyfoot[C]{\thepage}

% Configuración del documento
\begin{document}

% Portada
\begin{titlepage}
    \centering
    \vspace*{5cm}

    % Logo de la universidad
    \includegraphics[width=0.5\textwidth]{Imagenes/marca-uji-color.jpg}\par\vspace{1cm}

    {\Huge \bfseries GAME CONCEPT PRACTICE \par}
    {\large \bfseries Thy Ways are Dimnes \par}
    \vspace{1.5cm}

    % Información adicional
    \textsc{\large }
    \vspace{0.5cm} \\
    \textsc{\Large VJ1222}
    \vspace{0.5cm} \\
    \textsc{\large Year: 2024 - 2025}
    \vfill

    % Nombres de los autores
    \textbf{Made by:}         \\
    Jesús Jiménez Montero       \\

    \vspace{1cm}

    \vfill
\end{titlepage}

% % Abstract
% \begin{abstract}
%     Este documento de diseño "Ten Pages" presenta una visión general de los elementos clave de nuestro proyecto de videojuego, desde el arte hasta las mecánicas de juego. El proyecto está basado en el pueblo de \textit{Villanueva de Alcolea} y se desarrolla en el marco de las asignaturas VJ1222, VJ1223, y VJ1224. A lo largo de este documento, se detallan los aspectos fundamentales que conforman el conjunto del juego.
% \end{abstract}
% {\small \textbf{Palabras clave:} Triple, desarrollo, videojuego, pueblo.}
% \newpage

% Índice de contenidos
%change the table of contents title
\renewcommand{\contentsname}{Table of Contents}
\tableofcontents
\newpage

% Índice de figuras
% \listoffigures
% \newpage


\section{Game title.}
Thy Ways are Dimnes
\section{Public objective.}

\section{A summary of the game's story, focusing on gameplay.}
The game is set on a dark-medieval like world, where all human life has become extinct due to a event. This event was that the Earth crust collapsed, opening a big hole that opened the gates to dark creatures.

The creatures made of pure darkenss were mostly indestructible, except for a big weakness. Light and Fire; Light would expose them, and revealing their \textit{"flesh"} and being able to be damaged, then by using weapons covered in fire, these would be able to die.

And Humanity's will to live made them forcibly discover Gunpowder, which combined both elements into Humanity's last reasource.

Also, Humanity was able to persevere by moving their cities near the hole; as for some reason, as closer humanity was to the hole, the less the creatures would attack them. Thus, the villages near the Hole established and legion of Hunters was created.

As Gunpowder was limited, only the best warriors were assigned to the Gunpowder weapons; and tosave humanity only one thing can be done; go down to the Hole.

The player is one of those Hunters; and you need to kill everything down there and save them all.
\section{Distinct modes of gameplay.}

\section{Why this game is original?}

\section{For what reasons this game is interesting?}

\section{What are its main differences with video games of the same genre and what are this games?}


\end{document}
